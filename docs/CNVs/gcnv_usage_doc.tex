\documentclass[nofootinbib,amssymb,amsmath]{article}
\usepackage{mathtools}
\usepackage{amsthm}
\usepackage{algorithm}
\usepackage{algpseudocode}
\usepackage{lmodern}
\usepackage{graphicx}
\usepackage{color}
\usepackage{mathrsfs}
\usepackage{fullpage}
\usepackage{courier}

%Put an averaged random variable between brackets
\DeclareMathOperator*{\argmax}{\arg\!\max}
\newcommand{\ave}[1]{\left\langle #1 \right\rangle}

\newcommand{\vzero}{{\bf 0}}
\newcommand{\vI}{{\bf I}}
\newcommand{\vb}{{\bf b}}
\newcommand{\vd}{{\bf d}}
\newcommand{\vc}{{\bf c}}
\newcommand{\vv}{{\bf v}}
\newcommand{\vz}{{\bf z}}
\newcommand{\vn}{{\bf n}}
\newcommand{\vm}{{\bf m}}
\newcommand{\vG}{{\bf G}}
\newcommand{\vQ}{{\bf Q}}
\newcommand{\vM}{{\bf M}}
\newcommand{\vW}{{\bf W}}
\newcommand{\vX}{{\bf X}}
\newcommand{\vF}{{\bf F}}
\newcommand{\vZ}{{\bf Z}}

\newcommand{\vPsi}{{\bf \Psi}}
\newcommand{\vSigma}{{\bf \Sigma}}
\newcommand{\vlambda}{{\bf \lambda}}
\newcommand{\vLambda}{{\bf \Lambda}}
\newcommand{\vA}{{\bf A}}
\newcommand{\MM}{M}
\newcommand{\PP}{\mathcal{P}}
\newcommand{\EE}{\mathbb{E}}
\newcommand{\norm}{{\mathcal N}}

\newtheorem{lemma}{Lemma}
\newtheorem{corollary}{Corollary}

\def\SL#1{{\color [rgb]{0,0,0.8} [SL: #1]}}
\def\DB#1{{\color [rgb]{0,0.8,0} [DB: #1]}}

\newcommand{\HOM}{$\mathsf{Hom}$}
\newcommand{\HET}{$\mathsf{Het}$}
\newcommand{\REF}{$\mathsf{Ref}$}
\newcommand{\epss}{\varepsilon}

\DeclarePairedDelimiter\ceil{\lceil}{\rceil}
\DeclarePairedDelimiter\floor{\lfloor}{\rfloor}

\makeatletter
\DeclareFontFamily{OMX}{MnSymbolE}{}
\DeclareSymbolFont{MnLargeSymbols}{OMX}{MnSymbolE}{m}{n}
\SetSymbolFont{MnLargeSymbols}{bold}{OMX}{MnSymbolE}{b}{n}
\DeclareFontShape{OMX}{MnSymbolE}{m}{n}{
    <-6>  MnSymbolE5
   <6-7>  MnSymbolE6
   <7-8>  MnSymbolE7
   <8-9>  MnSymbolE8
   <9-10> MnSymbolE9
  <10-12> MnSymbolE10
  <12->   MnSymbolE12
}{}
\DeclareFontShape{OMX}{MnSymbolE}{b}{n}{
    <-6>  MnSymbolE-Bold5
   <6-7>  MnSymbolE-Bold6
   <7-8>  MnSymbolE-Bold7
   <8-9>  MnSymbolE-Bold8
   <9-10> MnSymbolE-Bold9
  <10-12> MnSymbolE-Bold10
  <12->   MnSymbolE-Bold12
}{}

\let\llangle\@undefined
\let\rrangle\@undefined
\DeclareMathDelimiter{\llangle}{\mathopen}%
                     {MnLargeSymbols}{'164}{MnLargeSymbols}{'164}
\DeclareMathDelimiter{\rrangle}{\mathclose}%
                     {MnLargeSymbols}{'171}{MnLargeSymbols}{'171}
\makeatother

\begin{document}

\title{GATK gCNV Usage Documentation}

\author{GATK CNV Team}

\date{\today}

\maketitle

\section{What is GATK gCNV?}
GATK gCNV is a read-depth based germline copy-number variant (CNV) calling tool.

\section{Overview of GATK gCNV pipeline}
The necessity of data modeling in general, ploidy determination, and data preparation in particular for GATK gCNV.

\subsection{COHORT mode}
todo.

\subsection{CASE mode}
todo.

\subsection{Scalability and task sharding}

\section{Setting up GATK gCNV runtime environment}
todo. setting up the \texttt{conda} environment for \texttt{gcnvkernel}.

\section{Preparing the data and resources}
todo.

\section{Components of GATK gCNV}

\subsection{\texttt{DetermineGermlineContigPloidy}}
todo.

\subsubsection{Purpose of the Tool}
todo.

\subsubsection{Operation Modes}
todo.

\subsubsection{Description of Arguments}
todo.

\subsection{\texttt{GermlineCNVCaller}}
todo.

\subsubsection{Purpose of the Tool}
todo.

\subsubsection{Operation Modes}
todo.

\subsubsection{Description of Arguments}
todo.

\subsection{\texttt{GermlineCNVPostProcessor}}
todo.

\subsection{Description of Hybrid ADVI Inference Arguments}
todo.

\section{Running GATK gCNV pipeline manually}
todo.

\section{Running GATK gCNV via WDL scripts}
todo.

\end{document}
